\documentclass[a4paper,11pt,titlepage]{article}
\usepackage[utf8]{inputenc} %support de l'utf8
\usepackage[T1]{fontenc} %support des accents
\usepackage[francais]{babel} %support de la langue
\usepackage{geometry}
\usepackage{listings}
\usepackage{color}
\usepackage{float} 
\usepackage{url}
\geometry{hscale=0.70,vscale=0.70,centering}
\usepackage{graphicx}
\usepackage{pstricks}
\usepackage{pst-tree}

\setcounter{secnumdepth}{4} % Augmente profondeur 
\setcounter{tocdepth}{4} % Augmente profondeur table des matieres
\makeatletter
\renewcommand{\paragraph}{\@startsection{paragraph}{4}{0ex}%
   {-3.25ex plus -1ex minus -0.2ex}%
   {1.5ex plus 0.2ex}%
   {\normalfont\normalsize\bfseries}}
\makeatother

\makeatletter
\def\url@biburlstyle{%
\@ifundefined{selectfont}{\def\UrlFont{\sf}}{\def\UrlFont{\small\ttfamily}}}
\makeatother

%\makeatletter
%\newcounter {subsubsubsection}[subsubsection]
%\renewcommand\thesubsubsubsection{\thesubsubsection .\@alph\c@subsubsubsection}
%\newcommand\subsubsubsection{\@startsection{subsubsubsection}{4}{\z@}%
%                                     {-3.25ex\@plus -1ex \@minus -.2ex}%
%                                     {1.5ex \@plus .2ex}%
%                                     {\normalfont\normalsize\bfseries}}
%\renewcommand\paragraph{\@startsection{paragraph}{5}{\z@}%
%                                    {3.25ex \@plus1ex \@minus.2ex}%
%                                    {-1em}%
%                                    {\normalfont\normalsize\bfseries}}
%\renewcommand\subparagraph{\@startsection{subparagraph}{6}{\parindent}%
%                          

%                                       {-1em}%
%                                      {\normalfont\normalsize\bfseries}}
%\newcommand*\l@subsubsubsection{\@dottedtocline{4}{10.0em}{4.1em}}
%\renewcommand*\l@paragraph{\@dottedtocline{5}{10em}{5em}}
%\renewcommand*\l@subparagraph{\@dottedtocline{6}{12em}{6em}}
%\newcommand*{\subsubsubsectionmark}[1]{}
%\makeatother

%%\usepackage{hyperref}

%%\makeatletter
%%\def\toclevel@subsubsubsection{4}
%%\def\toclevel@paragraph{5}
%%\def\toclevel@subparagraph{6}
%%\makeatother

\title{Rapport de Projet TEP\\Une applet d'évaluation des lambda-termes utilisant O'Browser}
\author{Thibault DUPERRON \and Nicolas RIGNAULT}



\begin{document}
\renewcommand{\bibname}{}
\renewcommand{\refname}{}
\maketitle
\tableofcontents

\newpage
\section{Introduction}

\subsection{Projet}

L'objectif est de réaliser un évaluateur de lambda-termes en utilisation OBrowser. Il s'agit donc d'implémenter les fonctions d'alpha-conversion et de béta-réduction en Ocaml et d'afficher le tout dans un navigateur Internet en utilisant OBrowser.

2 projets ont déjà été réalisés en PHP\footnote{http://stoufete.free.fr/TEP/page\_accueil.php} et en Java\footnote{http://www-apr.lip6.fr/~chaillou/Public/enseignement/2010-2011/tep/public/PROGS/AppletLG.html}.\\

Le but est donc d'implémenter le même comportement en utilisant OBrowser.


\subsection{OBrowser}

\subsubsection{Présentation}
OBrowser\footnote{http://www.pps.jussieu.fr/~canou/obrowser/tutorial/} est un interpreteur de bytecode Ocaml écrit en Javascript par Benjamin Canou. Le but est de pouvoir exécuter du code Ocaml dans un navigateur internet. Des fonctions permettant de manipuler des balises html ont donc été ajoutées afin d'être 
utilisées dans les programmes Ocaml.

\subsubsection{Tutoriel}
Les sources de OBrowser contiennent déjà certains exemples permettant de comprendre son fonctionnement. Voici un exemple supplémentaire de quelques outils de base.\\

On souhaite créer un champ de saisie de texte et un bouton. Un clique sur le bouton entrainerait l'affichage du contenu du champ de saisie de texte. On a donc besoin d'un bouton, d'un champ de saisie de texte et eventuellement d'un emplacement où ecrire le texte (par exemple une balise div).

Pour créer une balise, on dispose de ``Html.div''. Pour le bouton, on a ``Js.Node.element "input"''. Enfin on utilise un ``Node.text'' pour afficher une chaîne.\\

Le code correspondant à cet exemple se trouve en Annexe (Voir page \pageref{tutoObrowser}).


\subsection{Lambda-calcul}

\newpage
\section{Travail effecué}

\subsection{Parsing de chaîne de caractères}

\subsubsection{Objectif}
Le but de cette fonction est d'obtenir un arbre correspondant à la chaîne donnée.\\

Un type ``Term'' a été défini pour représenter les arbres pour le Lambda-calcul.
\begin{figure}[H]
\begin{center}
\begin{tabular}{c}

 \begin{lstlisting}
type term =
| Var of string
| Lambda of string * term
| App of term * term
 \end{lstlisting}

\end{tabular}
\end{center}
\caption{Type Term pour représenter les arbres pour le Lambda-calcul}
 \label{term}
\end{figure}
Ainsi, d'après ce type ``Term'' on souhaite donc obtenir un arbre. Ce dernier pourrait ainsi être facilement analysé pour pour pouvoir appliquer les fonctions nécessaires.
\\
\\\underline{Exemple:} L'expression (lx.xx)(lxy.xy) doit renvoyer l'arbre suivant

\begin{figure}[H]
\centering
  \psset{levelsep=1cm,nodesep=1mm}
  {\pstree{\Tr{App}}
	 {{\pstree{\Tr{Lambda x}}
		 {\pstree {\Tr{App}}
			  {
			   {\Tr{x}}
			   {\Tr{x}}
			  }
		 }
	 }
	{\pstree{\Tr{App}}
		 {\pstree {\Tr{Lambda x}}
			  {\pstree{\Tr{Lambda y}}
			   {{\Tr{x}}
			   {\Tr{y}}}
			  }
		 }
	}}
  }
\caption{Arbre correspondant à l'expression (lx.xx)(lxy.xy)}
 \label{tree1}
\end{figure}

\subsubsection{Algorithme}
L'algorithme utilisé est le suivant: la chaîne est analysée, caractère par caractère. Lorsque l'on recontre une parenthèse ouvrante, on cherche la prochaine parenthèse fermante correspondante. Puis, on relance la fonction sur le contenu de cette sous chaîne ainsi que sur le reste de la chaîne de départ. On constitut ainsi un sous arbre pour chaque chaine entre parenthèses. Ces chaines sont ensuites fusionnées dans l'ordre pour pouvoir obtenir l'arbre final.
Les différentes variables sont stockées dans des environnements avec leur attribut (savoir s'il s'agit d'un lambda par exemple).
\\
\\Exemple de décomposition pour la chaîne (lx.x(yz))(ly.y):
\begin{figure}[H]
\centering
  \psset{levelsep=1cm,nodesep=1mm}
  {\pstree{\Tr{(lx.x(yz))(ly.y)}}
	 {{\pstree {\Tr{(lx.x(yz))}}
			  {
			   {\Tr{(yz)}}
			   {\Tr{lx.x}}
			  }
		 
	 }
	{\Tr{(ly.y)}}
	}
  }
\caption{Arbre de décomposition de l'expression (lx.x(yz))(ly.y)}
 \label{tree2}
\end{figure}

Ici on commence donc par analyser la première expression : (lx.x(yz))\\
On stocke ``lx'' et ``x'' dans l'environnement et on analyse ensuite l'expression suivante ``yz''. Enfin on les fusionne.\\

On analyse ensuite l'expression suivante, (ly.y), on effectue le même traitement et on la fusionne avec l'expression précédente pour obtenir l'arbre final.
\begin{figure}[H]
\centering
  \psset{levelsep=1cm,nodesep=1mm}
  {\pstree{\Tr{App}}
	 {{\pstree{\Tr{Lambda x}}
		 {\pstree {\Tr{App}}
			  {
			   {\Tr{x}}
			   {\pstree {\Tr{App}}
			      {{\Tr{y}}
			       {\Tr{z}}}
			   }
			  }
		 }
	 }
	{\pstree{\Tr{Lambda y}}
		{\Tr{y}}


	}}
  }
\caption{Arbre correspondant à l'expression (lx.x(yz))(ly.y)}
 \label{tree3}
\end{figure}

\subsection{Implémentation Lambda-calcul}

\subsubsection{Choix du lambda à evaluer}

Le premier lambda à evaluer est celui qui est le plus à gauche dans l'arbre et dont le père est une Application.

Pour trouver ce sous arbre et sa position dans l'arbre d'origine, la fonction `rechercheLambda` fait un parcours recursif de l'arbre.
Pour chaque application, la fonction va regarder si un tel lambda existe dans le sous arbre gauche, si c'est le cas il renvoie le chemin vers ce lambda,
 sinon il regardera dans le sous arbre droit. Si aucun lambda n'est trouvé alors le chemin renvoyé est une liste vide.

\subsubsection{Alpha Conversion}


Les fonctions pour l'alpha conversion vont permettrent de modifier le nom des variables liées. En effet, dans le sous arbre trouvé par 
l'etape du choix du lambda à evaluer, on souhaite pouvoir renommer les variables liées sans modifier le sens de l'expression.\\

L'alpha conversion est faite en 3 étapes:
\begin{itemize}
 \item La fonction \textbf{listLambda} permet de lister tout les lambdas communs entre le sous arbre gauche et le sous arbre droit
 \item La fonction \textbf{listeNouveauNoms} permet, pour chaque lambda, de créer des couples (nom, nouveauNom) tel que le nouveau
nom de variable ne soit pas lié.
 \item La fonction \textbf{recreeArbreAlphaConv} permet de recréer un arbre en appliquant les modifications de noms de variables
\end{itemize}


\subsubsection{Beta Réduction}

  La fonction \textbf{recreeArbreBetaRed} permet grace aux informations données par l'étape du choix du lambda à evaluer d'effectuer la
beta réduction sur l'arbre.

  Pour cela, la fonction va laisser en l'état toutes les parties de l'arbre non concernées par la beta réduction, et va modifier
le sous arbre identifié précédement en remplacant chaque occurrence de la variable du lambda à réduire par le sous arbre
droit.


\subsection{Utilisation OBrowser}

  Nous avons choisis d'utiliser la librairie ``Graphics'' d'Ocaml car nous voulions afficher directement nos arbres dans le navigateur, ce qui est possible. Nous avons donc créé une page Html qui exécute le code Ocaml à l'ouverture via un script Javascript. Nous avons ensuite ajouté les éléments dont nous avions besoin, à savoir:
\begin{itemize}
 \item Un champ de saisie de texte pour récupérer l'expression de lambda-termes.
 \item Un bouton pour valider le texte et afficher la première etapes.
 \item Le graphe comportant notre arbre.
 \item Un bouton pour passer à la réduction suivante.
 \item Un bouton pour passer à l'étape précédente.
 \item Un texte affichant l'expression courante en fonction de l'arbre.
 \item Les différents exemples de bases\\
\end{itemize}

  Le code correspondant se trouve dans le fichier \textbf{evaluation.ml} et utilise les fonctions de parsing et d'operations sur les arbres.


\subsection{Manuel d'utilisation}

  Contenu de l'archive:
\begin{itemize}
 \item Code de l'évaluateur de Lambda-termes  -> \textbf{evaluation.ml}
 \item Code de la méthode d'analyse de chaîne -> \textbf{parsingstr.ml}
 \item Code des méthodes d'alpha-conversion et béta-réduction -> \textbf{operation.ml}
 \item Une page html exécutant le code compilé -> \textbf{test.html}
 \item Un répertoire contenant la documentation générées par Ocamldoc -> \textbf{doc}
 \item Un répertoire contenant les sources de OBrowser -> \textbf{obrowser}
 \item La machine virtuelle qui interprête le code OCaml -> \textbf{vm.js}
 \item Le présent rapport -> \textbf{rapport.pdf}\\
\end{itemize}

  Pour compiler le projet, il faut commencer par compiler les sources de obrowser en se rendant dans son répertoire (un Makefile est présent). Il reste ensuite à compiler les sources du projet (ici aussi un Makefile existe).\\

  Pour que le programme soit exécuté avec la page Html, il faut que la machine virtuelle soit présente. C'est déjà le cas dans notre archive mais elle peut être recompilée avec les sources obrowser.

\section{Annexes}
  \label{tutoObrowser}
  Code correspondant au tutoriel OBrowser:
  \begin{center}
  \begin{tabular}{c}

 \begin{lstlisting}
open Js ;; (* Necessaire *)
open Html ;; (* Necessaire si manipulation Html *)
open JSOO;; (* Necessaire pour la recuperation du contenu du bouton *)

(** Construit un champ de type input HTML pour la saisie de texte 
  @param name, le nom du champ
  @param value la valeur
  @return element
*)
let string_input name value =
  let input = Js.Node.element "input" in
  Js.Node.set_attribute input "type" "text" ;
  Js.Node.set_attribute input "value" value ;
  Js.Node.set_attribute input "style" value ;
input;;

(** Construit un bouton HTML 
  @param name, nom du bouton
  @param callback, fonction a appeler
*)
let button name callback =
  let input = Js.Node.element "input" in
  Js.Node.set_attribute input "type" "submit" ;
  Js.Node.set_attribute input "value" name ;
  Js.Node.register_event input "onclick" callback ();
input;;

(** Construit un element div HTML 
  @param id, l'id du DIV
*)
let div id =
  let div = Js.Node.element "div" in
  Js.Node.set_attribute div "id" id ;
div;;

(** Affiche le texte dans le div *)
let print () =
  let s = (get_element_by_id "text") >>> get "value" >>> as_string in
  let stringdiv = (get_element_by_id "string") in
  Node.empty stringdiv;
  Node.append stringdiv (Node.text s)
;;
 \end{lstlisting}

\end{tabular}
\end{center}
\newpage

  \begin{center}
  \begin{tabular}{c}

 \begin{lstlisting}
(** Fonction principale *)
let () = 

  (* Construit le div *)
  let stringdiv = Html.div ~style:"id: string; padding: 5px; 
    text-align: center; font-weight: bold; color: red;" [] in
  Js.Node.set_attribute stringdiv "id" "string" ;

  (* Construit le bouton *)
  let inputstring = string_input "Texte" "" in
  Js.Node.set_attribute inputstring "id" "text" ;	
	
  (* Ajoute les elements a la page *)
  let body = (get_element_by_id "body") in
  Node.append body (inputstring);
  Node.append body (button "Afficher"
  (fun () ->
    print ();
  ));
  Node.append body (stringdiv);;
 \end{lstlisting}

\end{tabular}
\end{center}

\end{document}