\documentclass[a4paper,11pt,titlepage]{article}
\usepackage[utf8]{inputenc} %support de l'utf8
\usepackage[T1]{fontenc} %support des accents
\usepackage[francais]{babel} %support de la langue
\usepackage{geometry}
\usepackage{listings}
\usepackage{color}
\usepackage{float} 
\usepackage{url}
\geometry{hscale=0.70,vscale=0.70,centering}
\usepackage{graphicx}

\setcounter{secnumdepth}{4} % Augmente profondeur 
\setcounter{tocdepth}{4} % Augmente profondeur table des matieres
\makeatletter
\renewcommand{\paragraph}{\@startsection{paragraph}{4}{0ex}%
   {-3.25ex plus -1ex minus -0.2ex}%
   {1.5ex plus 0.2ex}%
   {\normalfont\normalsize\bfseries}}
\makeatother

\makeatletter
\def\url@biburlstyle{%
\@ifundefined{selectfont}{\def\UrlFont{\sf}}{\def\UrlFont{\small\ttfamily}}}
\makeatother

%\makeatletter
%\newcounter {subsubsubsection}[subsubsection]
%\renewcommand\thesubsubsubsection{\thesubsubsection .\@alph\c@subsubsubsection}
%\newcommand\subsubsubsection{\@startsection{subsubsubsection}{4}{\z@}%
%                                     {-3.25ex\@plus -1ex \@minus -.2ex}%
%                                     {1.5ex \@plus .2ex}%
%                                     {\normalfont\normalsize\bfseries}}
%\renewcommand\paragraph{\@startsection{paragraph}{5}{\z@}%
%                                    {3.25ex \@plus1ex \@minus.2ex}%
%                                    {-1em}%
%                                    {\normalfont\normalsize\bfseries}}
%\renewcommand\subparagraph{\@startsection{subparagraph}{6}{\parindent}%
%                          

%                                       {-1em}%
%                                      {\normalfont\normalsize\bfseries}}
%\newcommand*\l@subsubsubsection{\@dottedtocline{4}{10.0em}{4.1em}}
%\renewcommand*\l@paragraph{\@dottedtocline{5}{10em}{5em}}
%\renewcommand*\l@subparagraph{\@dottedtocline{6}{12em}{6em}}
%\newcommand*{\subsubsubsectionmark}[1]{}
%\makeatother

%%\usepackage{hyperref}

%%\makeatletter
%%\def\toclevel@subsubsubsection{4}
%%\def\toclevel@paragraph{5}
%%\def\toclevel@subparagraph{6}
%%\makeatother

\title{Rapport de Projet TEP\\Une applet d'évaluation des lambda-termes utilisant O'Browser}
\author{Nicolas RIGNAULT \and Thibault DUPERRON}

\begin{document}
\renewcommand{\bibname}{}
\renewcommand{\refname}{}
\maketitle
\tableofcontents

\newpage
\section{Introduction}

\subsection{Présentation du lambda Calcul}

TODO

\subsection{Présentation d'Obrowser}

TODO\footnote{http://www.pps.jussieu.fr/~canou/obrowser/tutorial/}
\newpage

\section{Présentation des applet d'évaluation déjà existante}

\subsection{PHP}

TODO

\subsection{JAVA}

TODO

\newpage
\section{Travail effecué}

\subsection{Parsing de l'expression}

TODO

\subsection{Operations sur l'arbre}

\subsubsection{Choix du lambda à evaluer}

Le premier lambda à evaluer est celui qui est le plus à gauche dans l'arbre et dont le père est une Application.
listeNouveauNoms
Pour trouver ce sous arbre et sa position dans l'arbre d'origine, la fonction `rechercheLambda` fait un parcours recurcif de l'arbre.
Pour chaque application, la fonction va regarder si un tel lambda existe dans le sous arbre gauche, si c'est le cas il renvoi le chemin vers ce lambda,
 sinon il regardera dans le sous arbre droit. Si aucun lambda n'est trouvé alors le chemin renvoyé est une liste vide.

\subsubsection{Alpha Convertion}

Les fonctions pour l'alpha conversion vont permettrent de modifier le nom des variables liés dans le sous arbre trouvé par 
l'etape du choix du lambda à evaluer sans modifier le sens de l'expression.

L'alpha conversion est fait en 3 étapes:
\begin{itemize}
 \item La fonction listLambda permet de lister tout les lambdas communs entre le sous arbre gauche et le sous arbre droit
 \item La fonction listeNouveauNoms permet de donner pour chaque lambda de créer des couples (nom, nouveauNom) tel que le nouveau
nom de variable ne soit pas lié.
 \item La fonction recreeArbreAlphaConv permet de recréer un arbre en appliquant les modifications de noms de variables
\end{itemize}


\subsubsection{Beta Réduction}

  La fonction recreeArbreBetaRed permet grace aux informations données par l'etape du choix du lambda à evaluer d'effectuer la
beta réduction sur l'arbre.

  Pour cela, la fonction va laisser en l'état toutes les parties de l'arbre non concernées par la beta réduction, et va modifier
le sous arbre identifié précédement en remplacant chaque occurrence de la variable du lambda à réduire par le sous arbre
droit.

\end{document}
